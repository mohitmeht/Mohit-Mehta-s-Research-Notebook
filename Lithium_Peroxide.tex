\documentclass[Notes.tex]{subfiles}
\begin{document}
The energy density of Lithium air battery depends on porosity. The product which is of the utmost importance in these batteries is lithium peroxide. The various products form during discharge, \ce{Li2O2}, \ce{LiO2}, \ce{LiO}, \ce{Li2CO3}~\cite{Wang2012,Tan2015}.
The most prominent, most observed, and the most influential of these products is \ce{Li2O2}~\cite{Bardenhagen2015}. The formation of Lithium peroxide is the achiels heel of these batteries, not only is it the source of battery's high energy density but also a major cause for it's low practical energy density. Initial this product was thought to grow on the surface of the electrolyte until it covers all the pores on the air side of the cathode and block the oxygen from entering the system.
Recently it was observed by Shao-Yohn that the morphology of Li$_2$O$_2$ can change from a thin film to a toroidal shape. The toroid shape of the product is usually observed at low discharge currents and this form disappears at higher discharge currents (usually more than 1 mA/cm$^2$).  
\end{document}