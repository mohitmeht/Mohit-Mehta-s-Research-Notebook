\documentclass[Notes.tex]{subfiles}
\begin{document}
	
	
\chapter{EIS}

\section{History of EIS along with required conditions}
The full impedance spectra was not made until aprroximately 1970 with the advent of reliable potentiostat was easily available. The first technique which also became a de facto standard was Frequency Responce Analyzer (FRA) using single sine method, in which ac wave of 5-15 mV of a given frequency riding on a dc potential at desired frequencies. Important requirements of the system before EIS can be performed on it:
\begin{enumerate}
	\item Stable
	\item Reversible 
	\item System must be at equilibrium
\end{enumerate}
In 1985, Stoynov et al.~\cite{Stoynov1985} conducted an impedance study on non-stationary systems for the first time. Then in 1989, Chenebault et al.~\cite{Chenebault1989} measured the impedance of a Li metal electrode in \ce{LiAlCl4}/\ce{SOCl2} electrolyte while the system was under charging. Dynamic Electrochemical Impedance Spectroscopy (DEIS) was first applied to a lithium metal battery by Osaka et al., in 1994 and then much later it was applied to Li-ion battery in 2004~\cite{Itagaki2004,Itagaki2005}. DEIS has also been used in corrosion science and in fuel cell technology.
\section{Different EIS techniques}Stoynov \& Savova-Stoynov published mathematical expressions to extract instantaneous impedance values for nonstationary systems but it suffered from long acquisition time$^{\cite{Stoynov1985}}$.
EIS was also performed by applying noise signal and then demodulating the received signal. Fourier transform was also used it was known as first-generation Fourier transform EIS (1-FTEIS) by Smith$^{\cite{Smith1976}}$ and others. Rosvall \& Sharp used wavelet which consisted of 20 sinusoidal waves of 20 frequencies between 150 Hz and 200kHz riding on staircase voltammetry latter part of staircase$^{\cite{Rosvall2000}}$. Darowicki et al. performed cyclic voltammetry with EIS. They did this by overlapping wavelet of 7 signals on a slow ramp signal and they called this method as dynamic electrochemical impedance spectroscopy. This was used with lead acid batteries. 

\section{More EIS Background}
\paragraph*{June 18, 2014}
The diffusion term represented using the finite length Warburg impedance requires large number of RC circuits and it is balance between accuracy and performance (they implemented on a microcontroller), they suggested using 5 RC circuits in series to represent diffusion\cite{Fleischer2014}
The authors suggested the equation for charge transfer resistance for Li-ion batteries with polymer electrolytes.
\begin{align}
R_{ct}=R_\infty-R_{\textrm{mid}}            
\end{align}
Electrical vehicles can use battery packs with number of individual battery cells ranging from (20...200)\cite{Fleischer2014}.
If the symmetry factor $\alpha = 0,1$, the reactions can be thought to be irreversible. He also said that the oxidation number of reaction ions (n) can be non-integer. The paper was a theoretical paper with ECM modeling of supercapacitor and Li-ion\cite{Buller2005}.
\paragraph*{February 17, 2014}
The paper by Bonnefont and Argoul (Bazant) talks about AC current modulation on DC current~\cite{Bonnefont2000}. 

\section{Low frequency loop}
There is no consensus on the reason for the low frequency loop often observed on the impedance spectra on fuel cells operating with air.
The cited possible causes for the low frequency loop are diffusion of water through the "active layers"~\cite{Freire2001,Wagner1998} or though the membrane~\cite{Paganin1998}, oxygen diffusion through the "backing"~\cite{Springer1996}, gas flow rate through the cathode~\cite{Ciureanu2001}, and oxygen diffusion in nitrogen~\cite{Boillot2006}

\section{Applications of EIS}
\subsection{Calculation of Ionic Conductivity}
Hamon suggested a method to compute ionic conductivity using equivalent circuit modeling. Here the equation and an example is presented.
\begin{align}
\sigma= \frac{e}{R_{el}\times A}
\end{align}
where $e$ is the film thickness, $A$ is the area of the metal contact, and $R_{el}$ is the film thickness determined by the model.
\end{document}