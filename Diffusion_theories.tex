\documentclass[12pt, oneside]{article}
\usepackage{geometry}
\geometry{a4paper}
\usepackage{graphicx}
\usepackage{color}
\usepackage{enumerate}
\usepackage{hyperref}
\hypersetup{
    colorlinks,
    citecolor=black,
    filecolor=black,
    linkcolor=black,
    urlcolor=black
}
\begin{document}
\title{Diffusion based theories}
\author{Mohit Mehta}
\date{October 8, 2013}
\maketitle
Types of diffusion and diffusion based theories. This file will list out the different theories I have come across along with the references(In case you need to refer them later) 
\tableofcontents
\newpage
\section{Introduction}
Very good reference for diffusion theories is \textit{\url{http://sinnott.mse.ufl.edu/Backgrounds/theo02_diff.html}}

\section{Single File Diffusion}
Imagine a pipe which has particles with diameters half or greater than half. They move in as a single file or 1D. Number of cases were considered. 

\begin{enumerate}

\item {\bf{No jumping}}

Here the authors talk about the case where the particle cannot hop/jump/surpass another particle and their position is never changed with respect to the other particles.
\begin{figure}[htp!]
\centering 
{\includegraphics[width=0.5\textwidth,height=0.2\textheight]{figures/Single_file_diff_no_jump.pdf}} 
\caption{This figure gives an example of Single File Diffusion. The rectangular region represents a narrow channel with circular particles which cannot surpass other particle as defined by SFD.}
\label{fig:SFD_no_jump} 
\end{figure}
\newpage
\item {\bf{With jumping}}

Here the authors talk about the case where the particle can hop/jump/surpass another particle and their position does change however the center of mass or the group of molecules or particles still move together as a whole.
\begin{figure}[htp!]
\centering 
{\includegraphics[width=0.5\textwidth,height=0.2\textheight]{figures/Single_file_diff_with_jump.pdf}} 
\caption{This figure gives an example of Single File Diffusion. The rectangular region represents a narrow channel with circular particles which cannot surpass other particle as defined by SFD.}
\label{fig:SFD_with_jump} 
\end{figure}
\end{enumerate}
These models are explained in greater detail in ref. \cite{Hahn1998}
\newpage
\bibliographystyle{unsrt} 
\bibliography{library}
\end{document}